\documentclass[12pt]{article}

\usepackage{fullpage}
\usepackage{multicol,multirow}
\usepackage{tabularx}
\usepackage{listings}
\usepackage{color}
\usepackage[table]{xcolor}
\usepackage{tikz}
\usepackage{ulem}
\usepackage[russian]{babel}
\usepackage[utf8x]{inputenc}
\usepackage[T1, T2A]{fontenc}
\usepackage{fullpage}
\usepackage{multicol,multirow}
\usepackage{tabularx}
\usepackage{ulem}
\usepackage{tikz}
\usepackage{pgfplots}
\usepackage{indentfirst}

\begin{document}

\section*{Лабораторная работа №\,1 по курсу дискрeтного анализа: сортировка за линейное время}

Выполнила студентка группы М80-407Б-17 МАИ \textit{Алексюнина Юлия}.

\subsection*{Условие}
\begin{enumerate}
\item Требуется разработать программу, осуществляющую ввод пар «ключ-значение», их упорядочивание по возрастанию ключа алгоритмом сортировки подсчётом и вывод отсортированной последовательности. 
\item Вариант 3-3.

Поразрядная сортировка.

Тип ключа: числа от 0 до $2^{64}-1$.

Тип значения: числа от 0 до $2^{64}-1$.
\end{enumerate}

\subsection*{Метод решения}

\begin{enumerate}
    \item Данные на вход программе подаются через стандартный поток ввода и, как следствие, весьма удобно считывать циклом \textbf{while} (пока значение может быть прочитано, продолжать цикл). 
    \begin{lstlisting}[language=C++]
    while(std::cin >> el.Key >> el.Value)
    \end{lstlisting}
    \item По алгоритму сортировки необходимо знать максимальный ключ, он находится в цикле считывания данных.
    \item Собственно сам алгоритм сортировки принимает на вход вектор \textbf{v}, пары "ключ-значение"\ которого необходимо отсортировать по порядку возрастания ключей, и максимальный ключ. Также в коде сортировки присутствуют 2 дополнительных массива: массив \textbf{b[c[(v[i].Key)]]} для хранения отсортированных выходных данных, массив для временной работы \textbf{c[0...PART\_MASK]}, где \textbf{PART\_MASK} - максимальный ключ.
\end{enumerate}

\subsection*{Описание программы}

На каждой непустой строке входного файла располагается пара «ключ-значение»,
поэтому создадим новую структуру \textbf{Elem}, в которой будем хранить ключ и значение.

А также мы не знаем количество входных данных, поэтому мы напишем динамический массив - вектор, в который будут помещаться структуры 
\textbf{Elem}.
\begin{table}[!htb]
\begin{tabular}{|m{8cm}|m{8cm}|}
\hline
\multicolumn{2}{|c|}{Source.cpp} \\ 
\hline
\cellcolor{gray!25} Тип данных       & \cellcolor{gray!25} Значение\\ 
\hline
struct Elem & Структура для хранения пары "ключ-значение" \\ 
\hline
template <typename T> class Vector & Вектор для хранения структур \textbf{Elem}\\
\hline
\cellcolor{gray!25} Функция & \cellcolor{gray!25}Значение\\
\hline
Vector() & Конструктор класса \textbf{Vector} \\
\hline
Vector(int size, const value\_type\& default\_value = value\_type()) & Конструктор класса \textbf{Vector} \\
\hline
int Size() const & Размер вектора \\
\hline
friend void swap(Vector\& lhs, Vector\& rhs) & Обмен местами двух векторов\\
\hline
Vector\& operator=(Vector other) & Перегрузка оператора присваивания для класса \textbf{Vector}\\
\hline
$\sim$ Vector() & Деструктор класса \textbf{Vector} \\
\hline
void PushBack(const value\_type\& value) & Добавить элемент в конец вектора\\
\hline
void CountingSortBit(Vector <Elem>\& v, int size, int point) & Функция сортировки подсчетом\\
\hline
void RadixSort(Vector <Elem>\& v) & Функция поразрядной сортировки\\
\hline
int main() & Главная функция, в которой происходит чтение данных, вызов функции сортировки и вывод. \\
\hline
\end{tabular}
\end{table}

\newpage
\clearpage

\subsection*{Дневник отладки}

При создании этой таблицы была использована таблица локального репозитория.
\begin{table}[!htb]
\begin{tabular}{|m{2cm}|m{3cm}|m{11cm}|}
\hline
\multicolumn{3}{|c|}{Дневник отладки} \\ 
\hline
Время & Коммит & Описание \\
\hline
2020/10/25  19:47:27 & init & Начало работы, прочитаны главы из Кормена про алгоритмы сортировки, написаны шаблоны файлов и функций \\
\hline
2020/10/25 23:39:09 & parsing & продумывание возможных идей парсинга входных данных так, чтобы можно было работать с неизвестным количеством пар "ключ-значение" \\
\hline
2020/10/26 03:10:18 & сортировка & переписывание алгоритма сортировки по аналогии с алгоритмом, данным в Кормене. \\
\hline
2020/10/26 12:10:14 & сортировка, ч2 & дополнение алгоритма сортировки, организована побитовая сортировка подсчетом \\
\hline
2020/10/26 15:10:20 & рабочая версия программы & программа работает в vs, но рантаймится на чекере:( \\
\hline
2020/10/26 18:30:49 & optimization & поиск причин рантайма, оптимизация кода \\
\hline
2020/10/27 12:17:18 & death & создание класса Вектор для объектов пар "ключ-значение", прописывание основных полей и методов, оптимизация времени работы\\
\hline
2020/10/27 17:17:03 & alive & рабочая версия программы, ОК на чекере\\
\hline
\end{tabular}
\end{table}

\subsection*{Тест производительности}

Тесты создавались с помощью небольшой программы на языке C++:


\begin{lstlisting}[language=C++]
#include <iostream>
#include <fstream>
#include <ctime>
#include <cstdlib>

int main(int argc, char *argv[]) {
	std::ofstream file(argv[1]);
	srand(time(0));
	size_t size = atoi(argv[2]);
	char value[65];
	for (register int i = 0; i < size; i++) {
		file << rand() % 65535 << " ";
		for (register int j = 0; j < 64; j++) {
			value[j] = (char) (rand() % 26 + 97);
		}
		value[64] = '\0';
		file << value << "\n";
	}
	return 0;
}
\end{lstlisting}

\begin{tikzpicture}
	\begin{axis}[ylabel=Время,xlabel=Количество строк, width=15.5cm, height=10cm,grid=both]
	\addplot coordinates {
		( 100000, 1.170 )
		( 90000, 1.089 )
		( 80000, 0.944 )
		( 70000, 0.858 )
		( 60000, 0.727 )
		( 50000, 0.600 )
		( 40000, 0.469 )
		( 30000, 0.355  )
		( 20000, 0.262  )
		( 10000, 0.126  )};
	\addplot coordinates {
	( 10000, 0.126)
	( 100000, 1.170 )};
	\end{axis}

\end{tikzpicture}

\subsection*{Недочёты}

При неиспользовании опций std::ios\_base::sync\_with\_stdio(false)  и  std::cin.tie(NULL); программа не проходила тест по времени. После добавления этих опций время работы программы уменьшилось почти в 3 раза.

\subsection*{Выводы}

Данный алгоритм сортировки можно применять для обработки списка людей по году рождения, где ключ -- год рождения, а значение -- фамилия, имя, отчество.

В целом написание программы было полезным, посколько я получила опыт создания поразрядной сортировки, улучшила навыки отладки программы. 

\end{document}
